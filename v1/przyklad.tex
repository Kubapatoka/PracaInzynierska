% Opcje klasy 'iithesis' opisane sa w komentarzach w pliku klasy. Za ich pomoca
% ustawia sie przede wszystkim jezyk oraz rodzaj (lic/inz/mgr) pracy.
\documentclass[shortabstract]{iithesis}

\usepackage[utf8]{inputenc}

%%%%% DANE DO STRONY TYTUŁOWEJ
% Niezaleznie od jezyka pracy wybranego w opcjach klasy, tytul i streszczenie
% pracy nalezy podac zarowno w jezyku polskim, jak i angielskim.
% Pamietaj o madrym (zgodnym z logicznym rozbiorem zdania oraz estetyka) recznym
% zlamaniu wierszy w temacie pracy, zwlaszcza tego w jezyku pracy. Uzyj do tego
% polecenia \fmlinebreak.
\polishtitle    {Projektowanie i implementacja \fmlinebreak systemu magazynowego\fmlinebreak ze wspomaganiem procesu produkcji \fmlinebreak i zarządzaniem zasobami}
\englishtitle   {Design and implementation of a warehouse system \fmlinebreak supporting the production process and resource management}
\polishabstract {\ldots}
\englishabstract{\ldots}
% w pracach wielu autorow nazwiska mozna oddzielic poleceniem \and
\author         {Jakub Kopystiański}
% w przypadku kilku promotorow, lub koniecznosci podania ich afiliacji, linie
% w ponizszym poleceniu mozna zlamac poleceniem \fmlinebreak
\advisor        {dr Marcin Młotkowski}
%\date          {}                     % Data zlozenia pracy
% Dane do oswiadczenia o autorskim wykonaniu
%\transcriptnum {}                     % Numer indeksu
%\advisorgen    {dr. Jana Kowalskiego} % Nazwisko promotora w dopelniaczu
%%%%%

%%%%% WLASNE DODATKOWE PAKIETY
%
%\usepackage{graphicx,listings,amsmath,amssymb,amsthm,amsfonts,tikz}
%
%%%%% WŁASNE DEFINICJE I POLECENIA
%
%\theoremstyle{definition} \newtheorem{definition}{Definition}[chapter]
%\theoremstyle{remark} \newtheorem{remark}[definition]{Observation}
%\theoremstyle{plain} \newtheorem{theorem}[definition]{Theorem}
%\theoremstyle{plain} \newtheorem{lemma}[definition]{Lemma}
%\renewcommand \qedsymbol {\ensuremath{\square}}
% ...
%%%%%

\begin{document}

%%%%% POCZĄTEK ZASADNICZEGO TEKSTU PRACY

\chapter{Wprowadzenie}

\section{Motywacja}
	Motywacją do stworzenia takiego systemu jest realny problem z życia.  Prowadzę gospodarstwo rolne, którego głównym kierunkiem produkcji są jaja kurze. W związku z tym produkuję paszę dla kur niosek. Pasza składa się z około 10 różnych składników, a ich liczba jest uzależniona od dostępności niektórych surowców. Niektóre składniki takie jak pszenżyto, jęczmień, owies pochodzą z gospodarstwa, inne (np. śruta rzepakowa, kukurydza) są zamawiane z zewnętrznych firm. Cenną informacją jest to ile jakiego półproduktu jest obecnie w magazynie, aby przed przystąpieniem do procesu produkcji zgromadzić potrzebne materiały.

	Z takiego systemu również łatwo mógłbym wydedukować ile przykładowo pszenżyta zebrałem ze swoich pól . System będzie też w stanie obliczyć różnice między deklarowaną a rzeczywistą wielkością dostawy.
	
	System powinien również pomóc optymalizować koszty produkcji. 

\section {Moje oczekiwania od systemu}

\begin{itemize}
  \item Sprawdzanie ilości dostępnych półproduktów w magazynie
  \item Obliczanie rzeczywistych wielkości dostawy - występują różnice w deklarowanej masie oraz masie rzeczywistej
  \item Obliczanie wielkości zbiorów - Jeśli mam "dostawę" z mojego gospodarstwa o nieznanej masie mogę policzyć tę masę uwzględniająć zużycie zasobu z tej dostawy 
  \item  Wprowadzanie dostaw
  \item  Wprowadzanie strat
  \item  Obliczanie kosztów wykonania porcji paszy (produktu) w zależności od pochodzenia półproduktów
  \item  Wprowadzanie receptur na paszę
  \item  Automatyczne odejmowanie produktów ze stanu magazynowego podczas produkcji
\end{itemize} 

\section {Architektura}
	System będzie w formie aplikacji Web-owej. Planuję użyć frameworku Django oraz bazy danych PostgreSQL. Użyję również konteneryzacji (Docker)

\chapter{Plan pracy}
	\begin{enumerate}
		\item  Zapoznanie się z frameworkiem Django
			\begin{itemize}
				\item Wykonanie tutorialu 'Writing your first Django app', czyli utworzenie "basic poll aplication"
				\item Przegląd dostępych materiałów w serwisie Youtube w celu zdobycia praktycznych wskazówek przydatnych przy pisaniu programu.
			\end{itemize}
		\item Weryfikacja dotychczasowej wiedzy na temat produkcji pasz na podstawie książki "Zalecenia żywieniowe i wartość pokarmowa pasz dla drobiu" praca zbiorowa pod redakcją Stefanii Smulikowskiej i Andrzeja Rutkowskiego 2018.
		\item Zaplanowanie funkcjonalności programu
		\item Zaprojektowanie modeli, jest to równoznaczne z projektem bazy danych
		\item WIP
	\end{enumerate}
	
\chapter{Wiadomości z  książki Zalecenia żywieniowe i wartość pokarmowa pasz dla drobiu}

\section{Układ książki}
Zalecenia podzielone są na 15 rozdziałów i są opisane szczegółowo na ponad 130 stronach. 
Pierwsze trzy rozdziały zawierają wiedzę ogólną na temat składników materiałów paszowych.
Rozdziały od 4. do 13. włącznie są szczegółowymi zaleceniami dla różnych gatunków drobiu ("Zalecenia żywienia perlic", "Zalecenia żywienia kur", "Zalecenia żywienia gęsi").
Rozdział 14-sty "Zawartość składników pokarmowych w materiałach paszowych w żywieniu drobiu" zawiera uporządkowane w tabelach informacje.
Postaram się streścić najważniejsze informacje w kontekście bilansowania mieszanki paszowej.

\section{Wilgotność}
W tabelach określa się zawartości poszczególnych składników pokarmowych przy założeniu $12\%$ wilgotności. Jest to częsta wilgotność w jakiej przechowuje się zboża. 
Jeśli zawartość suchej masy w kilogramie materiału paszowego jest inna niż w tabeli (zazwyczaj w tabeli widnieje 880g/kg), zawartość składników pokarmowych należy przeliczyć zgodnie ze wzorem: 

$$ S(g/kg) = S(tabela) * \frac{SuchaMasaNowa}{Sucha Masa Tabela}  $$
 
 
 \section{Energia}
 Zawartość energii jest wyrażana w jednostkach pozornej energii metabolicznej - $EM_n$. Jest to różnica energii brutto (ciepło spalania) i energii odchodów (kałomoczu w przypadku drobiu) i gazów.
 Energię tradycyjnie mierzono i wyrażano w kaloriach, ale obecnie powinniśmy stosować się do Międzynarodowego Układu Jednostek Miar i wyrażać tę energię w dżulach. 
 W opisywanym opracowaniu energia jest wyrażana zarówno w kilokaloriach (kcal) jak i megadżulach (MJ)
 
$$1 kcal = 0,004184 MJ$$
$$1 MJ = 239 kcal$$ 

Potrzeby energetyczne są określone dla każdego gatunku i stadium rozwoju, ponieważ różnią się między sobą. Jeśli mieszanka paszowa będzie miała niedobór energii, objawi się to większym spożyciem paszy.
Natomiast z nadmiarem energii w paszy idzie w parze otłuszczenie ptaków.

\section{Białko i Aminokwasy}
W białkach ciała i jaj ptaków znajdują się $22$ aminokwasy, które dzielimy na trzy kategorie:
\begin{itemize}
\item aminokwasy niezbędne (egzogenne)
\item aminokwasy względnie egzogenne
\item aminokwasy nie-niezbędne (endogenne)
\end{itemize}

Aminokwasy niezbędne muszą być dostarczane w paszy w odpowiednich proporcjach. Najlepiej zbadane jest zapotrzebowanie na te aminokwasy u kurczątk brojlerów i indyków. 
W omawianym opracowaniu dla każdego gatunku podano zapotrzebowanie tylko na te aminokwasy egzogenne, których niedobór może wystąpić w mieszankach paszowych tradycyjnych.
Na pozostałych aminokwasach nie ma potrzeby skupiania swojej uwagi przez zależność: Jeśli poprzednie aminokwasy są w wystarczających ilościach, to pozostałe również.

\subsection{Aminokwasy niezbędne (egzogenne)}
Aminokwasy egzogenne to takie których ptaki nie mogą syntetyzować. Są to dokładnie:
\begin{itemize}
\item lizyna
\item metionina
\item treonina
\item tryptofan
\item arginina
\item fenyloalanina
\item histydyna
\item izoleucyna
\item leucyna
\item walina
\end{itemize}

\subsection{Aminokwasy względnie egzogenne}
Te aminokwasy są względnie syntetyzowane w organizmie z aminokwasów egzogennych np. cysteina z metioniny.

\subsection{Aminokwasy nie-niezbędne (endogenne)}
Aminokwasy endogenne są syntezowane w organizmie. Są to takie aminokwasy jak:
\begin{itemize}
\item alanina
\item asparagina
\item glutamina
\item kwas asparaginowy
\item kwas glutaminowy
\item prolina
\item seryna
\item hydroksyprolina
\item glicyna
\end{itemize}


\section{Tłuszcz}
Dodatek tłuszczu pozwala na zwiększenie wartości energetycznej mieszanki paszowej, aby wyrównać ją do zapotrzebowania na energię.
Oprócz tego ptaki potrzebują określonej ilości tzw. niezbędnych nienasyconych kwasów tłuszczowych (NNKT). Są to kwasy linolowy i kwas $\alpha-$linolenowy.
W tabelach występuje zapotrzebowanie tylko na kwas linolowy, ponieważ zapotrzebowanie na kwas $\alpha-$linolenowy jest około 10-krotnie mniejsze.


\section{Składniki mineralne}

\subsection{Wapń}


 

\section{Przykładowa pasza}

	Parametry do skomponowania przykładowej$^*$ mieszanki paszowej to:
	
	\begin{tabular}{|c|c|}
		\hline
		Lizyna &0,698 \\
		\hline
		Metionina &0,344 \\
		\hline
		Metionina + cystyna &0,629 \\
		\hline
		Treonina &0,551 \\
		\hline
		Tryptofan &0,147 \\
		\hline
		Arginina &0,732 \\
		\hline
		Walina &0,603 \\
		\hline
		Izoleucyna &0,569 \\
		\hline
		Wapń &3,110 \\
		\hline
		Fosfor przyswajalny &0,319 \\
		\hline
		Sód &0,137 \\
		\hline
	\end{tabular}
	
	Dane wyrażone są w jednostce g/$1$ MJ EM.
	
	Poziom EM w paszy powinien wynosić $11,5-11,7$ MJ/kg paszy
	
	
	* - Dla grupy wiekowej kur niosek która dopiero co weszła w okres nieśności (od około $5\%$ nieśności w stadzie do około 45 tygodnia życia)
	


\chapter{Funkcje systemu}
	\begin{itemize}
  		\item  Wprowadzanie dostaw
  		\item  Wprowadzanie strat
  		\item  Automatyczne odejmowanie produktów ze stanu magazynowego podczas produkcji
  		\item  Sprawdzanie ilości dostępnych półproduktów w magazynie
  		\item  Obliczanie rzeczywistych wielkości dostawy - występują różnice w deklarowanej masie oraz masie rzeczywistej
  		\item  Obliczanie kosztów wykonania porcji paszy (produktu) w zależności od pochodzenia półproduktów
  		\item  Komponowanie receptury na paszę, w oparciu o produkty dostępne w magazynie - optymalizacja kosztów
  		\item  Wprowadzanie receptur na paszę
	\end{itemize} 


\chapter{Projektowanie Modeli}

\section{Dostawa}
Podstawową jednostką w systemie jest dostawa. To dostawy będą zapisane w bazie danych i na ich podstawie będą oblczane stany magazynowe.
Dostawa składa się z:
\begin{itemize}
	\item wskazanie na produkt
	\item cena
	\item data
	\item początkowa ilość
	\item zużyta ilość
	\item strata
	\item wartość boolowska czy dostawa się zakończyła
\end{itemize}

Jeśli dostawa jest zakończona to powinien zachodzić warunek $$strata + zużyta ilość = początkowa ilość$$

\section{Produkt}
Nie wyjaśniłem jeszcze czym jest wskazanie na produkt i czym jest produkt. Produkt jest zdefiniowany jako lista parametrów oraz nazwa.
Lista znaczących parametrów w produkcji paszy dla drobiu według "Zalecenia żywieniowe i wartość pokarmowa pasz dla drobiu" to:
\begin{itemize}
	\item 	Wilgotność,
	\item	Lizyna, 
	\item	Metionina,
	\item 	Metionina+cystyna, 
	\item	Treonina, 
	\item	Tryptofan, 
	\item	Arginina, 
	\item	Walina, 
	\item	Izoleucyna, 
	\item	Białko ogólne, 
	\item	Wapń, 
    	\item 	Fosfor przyswajalny, 
    	\item	Sód, 
    	\item	Kwas Linolowy 
\end{itemize}

Tutaj pojawiła się opcja rozszerzenia możliwości systemu o inne parametry. Przykładowo inne białka i właściwości danych produktów mogą odgrywać znaczącą rolę w produkcji pasz dla bydła, trzody lub hodowli ryb. Póki co zdecydowałem wpisać wymienione wyżej parametry "na sztywno".

\subsection{Przypadek użycia}
Gospodarz posiał dwa pola Łubinu żółtego, odmianę "Mister" na działce numer $12$ oraz odmianę "Salut" na działce numer $72$. Po zbiorach Łubinu żółtego gospodarz wysyła próbkę z każdego pola do badań w laboratorium. Otrzymuje następujące wyniki:

\begin{tabular}{|c|c|c|}
\hline
cecha &Mister &Salut \\
\hline
\end{tabular}





%%%%% BIBLIOGRAFIA

%\begin{thebibliography}{1}
%\bibitem{example} \ldots
%\end{thebibliography}

\end{document}
