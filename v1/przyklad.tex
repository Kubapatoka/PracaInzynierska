% Opcje klasy 'iithesis' opisane sa w komentarzach w pliku klasy. Za ich pomoca
% ustawia sie przede wszystkim jezyk oraz rodzaj (lic/inz/mgr) pracy.
\documentclass[shortabstract]{iithesis}

\usepackage[utf8]{inputenc}

%%%%% DANE DO STRONY TYTUŁOWEJ
% Niezaleznie od jezyka pracy wybranego w opcjach klasy, tytul i streszczenie
% pracy nalezy podac zarowno w jezyku polskim, jak i angielskim.
% Pamietaj o madrym (zgodnym z logicznym rozbiorem zdania oraz estetyka) recznym
% zlamaniu wierszy w temacie pracy, zwlaszcza tego w jezyku pracy. Uzyj do tego
% polecenia \fmlinebreak.
\polishtitle    {Projektowanie i implementacja \fmlinebreak systemu magazynowego\fmlinebreak ze wspomaganiem procesu produkcji \fmlinebreak i zarządzaniem zasobami}
\englishtitle   {Design and implementation of a warehouse system \fmlinebreak supporting the production process and resource management}
\polishabstract {\ldots}
\englishabstract{\ldots}
% w pracach wielu autorow nazwiska mozna oddzielic poleceniem \and
\author         {Jakub Kopystiański}
% w przypadku kilku promotorow, lub koniecznosci podania ich afiliacji, linie
% w ponizszym poleceniu mozna zlamac poleceniem \fmlinebreak
\advisor        {dr Marcin Młotkowski}
%\date          {}                     % Data zlozenia pracy
% Dane do oswiadczenia o autorskim wykonaniu
%\transcriptnum {}                     % Numer indeksu
%\advisorgen    {dr. Jana Kowalskiego} % Nazwisko promotora w dopelniaczu
%%%%%

%%%%% WLASNE DODATKOWE PAKIETY
%
%\usepackage{graphicx,listings,amsmath,amssymb,amsthm,amsfonts,tikz}
%
%%%%% WŁASNE DEFINICJE I POLECENIA
%
%\theoremstyle{definition} \newtheorem{definition}{Definition}[chapter]
%\theoremstyle{remark} \newtheorem{remark}[definition]{Observation}
%\theoremstyle{plain} \newtheorem{theorem}[definition]{Theorem}
%\theoremstyle{plain} \newtheorem{lemma}[definition]{Lemma}
%\renewcommand \qedsymbol {\ensuremath{\square}}
% ...
%%%%%

\begin{document}

%%%%% POCZĄTEK ZASADNICZEGO TEKSTU PRACY

\chapter{Wprowadzenie}

\section{Motywacja}
	Motywacją do stworzenia takiego systemu jest realny problem z życia.  Prowadzę gospodarstwo rolne, którego głównym kierunkiem produkcji są jaja kurze. W związku z tym produkuję paszę dla kur niosek. Pasza składa się z około 10 różnych składników, a ich liczba jest uzależniona od dostępności niektórych surowców. Niektóre składniki takie jak pszenżyto, jęczmień, owies pochodzą z gospodarstwa, inne (np. śruta rzepakowa, kukurydza) są zamawiane z zewnętrznych firm. Cenną informacją jest to ile jakiego półproduktu jest obecnie w magazynie, aby przed przystąpieniem do procesu produkcji zgromadzić potrzebne materiały.

	Z takiego systemu również łatwo mógłbym wydedukować ile przykładowo pszenżyta zebrałem ze swoich pól . System będzie też w stanie obliczyć różnice między deklarowaną a rzeczywistą wielkością dostawy.
	
	System powinien również pomóc optymalizować koszty produkcji. 

\section {Moje oczekiwania od systemu}

\begin{itemize}
  \item Sprawdzanie ilości dostępnych półproduktów w magazynie
  \item Obliczanie rzeczywistych wielkości dostawy - występują różnice w deklarowanej masie oraz masie rzeczywistej
  \item Obliczanie wielkości zbiorów - Jeśli mam "dostawę" z mojego gospodarstwa o nieznanej masie mogę policzyć tę masę uwzględniająć zużycie zasobu z tej dostawy 
  \item  Wprowadzanie dostaw
  \item  Wprowadzanie strat
  \item  Obliczanie kosztów wykonania porcji paszy (produktu) w zależności od pochodzenia półproduktów
  \item  Wprowadzanie receptur na paszę
  \item  Automatyczne odejmowanie produktów ze stanu magazynowego podczas produkcji
\end{itemize} 

\section {Architektura}
	System będzie w formie aplikacji Web-owej. Planuję użyć frameworku Django oraz bazy danych PostgreSQL. Użyję również konteneryzacji (Docker)

\chapter{Plan pracy}
	\begin{enumerate}
		\item  Zapoznanie się z frameworkiem Django
			\begin{itemize}
				\item Wykonanie tutorialu 'Writing your first Django app', czyli utworzenie "basic poll aplication"
				\item Przegląd dostępych materiałów w serwisie Youtube w celu zdobycia praktycznych wskazówek przydatnych przy pisaniu programu.
			\end{itemize}
		\item Weryfikacja dotychczasowej wiedzy na temat produkcji pasz na podstawie książki "Zalecenia żywieniowe i wartość pokarmowa pasz dla drobiu" praca zbiorowa pod redakcją Stefanii Smulikowskiej i Andrzeja Rutkowskiego 2018.
		\item Zaplanowanie funkcjonalności programu
		\item Zaprojektowanie modeli, jest to równoznaczne z projektem bazy danych
		\item WIP
	\end{enumerate}
	
\chapter{Wiadomości z  książki "Zalecenia żywieniowe i wartość pokarmowa pasz dla drobiu"}
	Parametry do skomponowania idealnej mieszanki paszowej to:
	\begin{tabular}{|c|c|}

\chapter{Projektowanie Modeli}




%%%%% BIBLIOGRAFIA

%\begin{thebibliography}{1}
%\bibitem{example} \ldots
%\end{thebibliography}

\end{document}
